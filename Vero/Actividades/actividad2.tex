% Lenguajes de Programación 2019-1
% Plantilla para reportes de laboratorio.

\documentclass[spanish,12pt,letterpaper]{article}

\usepackage[spanish]{babel}
\usepackage[utf8]{inputenc}
\usepackage{authblk}
\usepackage{listings}
\usepackage{dsfont}
\usepackage{lscape}
\usepackage{multirow}
\title{Actividad II}
\author{Emiliano Galeana Araujo}
\affil{Facultad de Ciencias, UNAM}
\date{Fecha de entrega: 18 de Febrero de 2019}

\begin{document}

\maketitle

Lo primero que hice fue escribir las pistas indivuales, por ejemplo El inglés
vive enn la casa roja, etc. Luego a todas las demás casillas les ponía la
negación, por ejemplo siguiendo la pista anterior, sabríamos que el japonés,
noruego, español ni ucraniano tendrían la casa roja.\\
Luego usé las pistas ``dobles'' como la de en la casa verde se toma café, aún no
tenía quién tenía la casa verde, pero sabía que la roja no tomaba café, así lo
mismo para las demás pistas.\\
Finalmente con las pistas ``triples'' lo que hice fue ubicar por ejemplo que el
noruego vive en la primer casa y es de color amarilla.\\
Luego intenté hacer resoluciones con algunas pistas para ir rellenando mi
cuadrícula, e ir avanzando; Pero hubo una parte donde me trabé, ya no tenía otra
pista u algo que me dijera. Analicé los cuadros faltantes y vi que las bebidas
estaban casi completas, noté que el inglés tenía todas sus bebidas negadas, así
que solo había dos opciones, o bebe leche o bebe jugo. Supuse que bebe leche y
así fui rellenando los demás cuadros; Sabía que si llegaba a una contradicción,
por ejemplo (No me acuerdo exactamente de la cuadrícula) que si el inglés bebía
leche, me llevaba a X, y X a que el ucraniano fumaba old gold es una
contradicción.\\
Afortunadamente estuvo bien la suposición y el inglés si bebe leche, y no llegué
a ninguna y acabé el acertijo.\\\\
Notas: No usé la página que nos diste.\\\\
Respuestas:\\
\begin{itemize}
\item El noruego bebe agua.
\item La zebra pertenece al japonés.
\end{itemize}
\begin{table}[htbp]
  \begin{center}
    \begin{tabular}{|l|l|l|l|l|l|}
      \hline
      & Casa & Mascota & Bebida & Cigarros & Posición \\ \hline
      & \underline{roja} &$\neg$ perro &$\neg$ café &$\neg$ kools &$\neg$ 1\\ 
      &$\neg$ verde &$\neg$ caballo &$\neg$ té &$\neg$ parliaments &$\neg$ 2\\ 
      Inglés    &$\neg$ azul &$\neg$ zebra &$\neg$ agua &$\neg$ Lucky &\underline{3} \\
      &$\neg$ amarilla &\underline{caracoles} &\underline{leche} &\underline{old gold} &$\neg$ 4 \\ 
      &$\neg$ marfil &$\neg$ zorro &$\neg$ jugo &$\neg$ chesterfields &$\neg$ 5 \\ \hline
      &$\neg$ roja &$\neg$ perro &$\neg$ te & \underline{parliaments} &$\neg$ 1 \\
      &$\neg$ amarilla &$\neg$ caracoles &$\neg$ jugo &$\neg$ Lucky &$\neg$ 3 \\ 
      Japonés   &\underline{verde} &$\neg$ caballo &$\neg$ agua &$\neg$ kools &\underline{5} \\
      &$\neg$ azul &$\neg$ zorro &\underline{café} &$\neg$ old gold&$\neg$ 2\\
      &$\neg$ marfil &\underline{zebra} &$\neg$leche &$\neg$ chesterfields &$\neg$ 4\\ \hline
      &$\neg$ roja &$\neg$ perro &$\neg$ té &$\neg$ parliaments &\underline{1}\\
      &$\neg$ azul&$\neg$ caballo &$\neg$ leche &\underline{kools} &$\neg$2\\
      Noruego   &$\neg$ verde &$\neg$ zebra &$\neg$ café &$\neg$ Lucky &$\neg$3\\
      &$\neg$ marfil &$\neg$ caracoles &$\neg$ jugo &$\neg$ chesterfields &$\neg$ 4\\
      &\underline{Amarilla} &\underline{zorro} &\underline{agua} &$\neg$ old gold &$\neg$ 5\\\hline
      &$\neg$ roja &\underline{perro} &$\neg$ té &$\neg$ old gold &$\neg$ 1\\
      &$\neg$ amarilla &$\neg$zorro &$\neg$ agua &$\neg$ parliaments &$\neg$ 2\\
      Español   &$\neg$ azul &$\neg$ caracoles &\underline{jugo} &$\neg$ kools &$\neg$ 3\\
      &$\neg$ verde &$\neg$ zebra &$\neg$ leche &\underline{Lucky} &\underline{4}\\
      &\underline{marfil} &$\neg$ caballo&$\neg$ café&$\neg$ chesterfields&$\neg$ 5\\\hline
      &$\neg$ roja &$\neg$ perro &\underline{té} &$\neg$ Lucky &$\neg$ 3\\
      &$\neg$ verde&\underline{caballo} &$\neg$ agua &$\neg$ parliaments &$\neg$ 1\\
      Ucranuano &$\neg$ amarilla &$\neg$ zebra &$\neg$café &$\neg$ kools &\underline{2}\\
      &$\neg$ marfil &$\neg$ caracoles &$\neg$ jugo&$\neg$ old gold &$\neg$ 4\\
      &\underline{azul} &$\neg$ zorro &$\neg$ leche&\underline{chesterfields}&$\neg$ 5\\\hline
    \end{tabular}
    \caption{Cuadro de valores del acertijo.}
    \label{tabla:sencilla}
  \end{center}
\end{table}

\end{document}